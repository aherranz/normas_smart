\documentclass[12pt]{article}

\usepackage[a4paper,margin=30mm]{geometry}

\usepackage[T1]{fontenc}
\usepackage[utf8]{inputenc}

\usepackage{titling}
\usepackage{parskip}

\usepackage{fourier}
% \usepackage[scaled]{berasans}
% \renewcommand*\familydefault{\sfdefault}  %% Only if the base font of the document is to be sans serif


\usepackage[spanish]{babel}

\usepackage{csquotes}

\title{Normas para el uso de mi móvil}
\author{v1.1.0}
\date{24 de octubre de 2019}

\begin{document}

\setlength{\droptitle}{-25mm}

\maketitle

Todas estas \textbf{normas están para proteger mi salud}, física y
mental, \textbf{mi relación con las personas} a las que quiero y me quieren, y
\textbf{mis estudios}. La mayoría de estas normas no aplican sólo a al
móvil, también \textbf{aplican a cualquier otro dispositivo con acceso
  a Internet}, como por ejemplo una \emph{tablet} o un ordenador.

%%%%%%%%%%%%%%%%%%%%%%%%%%%%%%%%%%%%%%%%%%%%%%%%%%%%%%%%%%%%%%%%%%%%%%%%
\begin{center}
  \textsc{Sobre el móvil}
\end{center}

\begin{enumerate}

\item El móvil \textbf{no es mío}, es de mis padres, tengo \textbf{la
    obligación de cuidarlo} y el \textbf{derecho a hacer un uso
    responsable} del mismo.

\item Tener un móvil, especialmente un \emph{smartphone}, \textbf{es un
    privilegio}, tengo la \textbf{obligación de valorarlo}
    adecuadamente.

\item Soy el \textbf{responsable de lo que le ocurra al
    móvil}. \textbf{Cuidaré} del móvil para evitar que se golpee,
  aplaste, moje o se pierda. Mis padres \textbf{no podrán comprar otro}.

%%%%%%%%%%%%%%%%%%%%%%%%%%%%%%%%%%%%%%%%%%%%%%%%%%%%%%%%%%%%%%%%%%%%%%%%
\begin{center}
  \textsc{Sobre los contenidos}
\end{center}

\item Mis padres \textbf{conocerán todas las contraseñas} de acceso.

\item \textbf{No puedo darme de alta} en ningún servicio,
  especialmente en una red social, sin hablar primero con mis
    padres.

% \item \textbf{No puedo instalar} ninguna App en el móvil sin
%     hablar primero con mis padres.

\item \textbf{No puedo descargar} contenidos sin hablar
    primero con mis padres.

\item \textbf{No puedo hacer compras} con el móvil.

\item \textbf{Evitaré acceder a contenidos inapropiados para mi edad}
  como porno, violencia, lenguaje soez, etc.

\item \textbf{Hablaré con mis padres} si sé que alguien sube contenido inapropiado.

% \item De vez en cuando escucharé \textbf{\textquote{otro tipo} de
%     música}.

%%%%%%%%%%%%%%%%%%%%%%%%%%%%%%%%%%%%%%%%%%%%%%%%%%%%%%%%%%%%%%%%%%%%%%%%
\begin{center}
  \textsc{Sobre los horarios}
\end{center}

\item Cuando al día siguiente tenga colegio, \textbf{apagaré el móvil a las
  22:00}, lo dejaré \textbf{fuera de mi habitación}, y no volveré a encenderlo
  hasta que esté \textbf{preparado para salir por la puerta} a la mañana
  siguiente.

\item Cuando no tenga colegio al día siguiente, \textbf{apagaré el móvil a las
  23:30}, lo dejaré \textbf{fuera de mi habitación}, y no volveré a encenderlo
  \textbf{hasta que haya desayunado} al día siguiente.

\item \textbf{En general}, podré llevar el móvil \textbf{para estar
    localizable} antes de entrar y al salir. \textbf{El móvil estará apagado} mientras esté en horario de clases.

\item Si llevo el móvil al instituto \textbf{respetaré las normas del
  instituto}. Además de que mis padres y profesores pueden castigarme,
  también puedo poner en riesgo la libertad de mis compañeros.

%%%%%%%%%%%%%%%%%%%%%%%%%%%%%%%%%%%%%%%%%%%%%%%%%%%%%%%%%%%%%%%%%%%%%%%%
\clearpage
\begin{center}
  \textsc{Sobre lo que digo y subo}
\end{center}

\item \textbf{No diré} ni escribiré nada a través del móvil
  \textbf{que pueda hacer daño a otros}.

\item Cuando quiera decir o escribir algo a través del móvil,
  especialmente si estoy nervioso o enfadado, \textbf{me tomaré un
    momento} para considerar si \textbf{lo diría en persona}.

\item \textbf{Me tomaré un momento} para considerar si algo de lo que haga con
  el móvil puede \textbf{herir, molestar o avergonzar} a mis padres, a mi mismo o a otras personas.

\item No sacaré ni difundiré \textbf{contenidos feos, de mal gusto, o
    comprometidos}, ni míos ni de nadie. Cuando envío una foto o un vídeo personal \textbf{ya no tengo control} sobre ese contenido y estará \textbf{en
  Internet para el resto de mi vida}. Además, está \textbf{prohibido por la ley difundir} fotos o videos de otros. Si tengo dudas con un contenido
  puedo consultárselo a mis padres.

%%%%%%%%%%%%%%%%%%%%%%%%%%%%%%%%%%%%%%%%%%%%%%%%%%%%%%%%%%%%%%%%%%%%%%%%
\begin{center}
  \textsc{Sobre el uso}
\end{center}

\item \textbf{Si recibo una llamada} de un número de la familia, especialmente
  \textbf{de mis padres}, \textbf{debo contestar}.

\item \textbf{No contestaré} a llamadas de números \textbf{desconocidos}.

\item \textbf{No} entraré con el móvil \textbf{al baño}.

\item \textbf{No} usaré el móvil \textbf{en la mesa}, durante el desayuno, la comida, o la cena.

\item \textbf{Evitaré que mi uso del móvil moleste a otras
    personas}. Esto incluye, por ejemplo, poner la música demasiado
  alta o usarlo en el cine.

\item \textbf{Silenciaré} el móvil y lo dejaré \textbf{a un lado}
  cuando esté en un \textbf{restaurante}, en el \textbf{cine} o \textbf{hablando} con otra
  persona.

\item \textbf{Debo estar atento} a lo que otros puedan hacer con mi
  móvil \textbf{si lo presto}. Alguien podría suplantar mi identidad
  en Internet.

\item \textbf{Me autocontrolaré} en el uso del móvil. El uso del móvil puede
  provocarme enfermedades o problemas que pueden durar toda mi vida:
  miopía, problemas de espalda, adicción, etc.

\item Dejaré el \textbf{móvil en casa de vez en cuando}. Ser capaz de
  hacerlo es una forma de saber que no me he convertido en
  un adicto.

\item \textbf{Si tengo dudas} sobre si puedo usar el móvil de una determinada
  forma me haré la siguiente pregunta y pensaré en la respuesta:
  \textbf{\textquote{¿Qué pasaría si todos hiciéramos lo mismo?}}

%%%%%%%%%%%%%%%%%%%%%%%%%%%%%%%%%%%%%%%%%%%%%%%%%%%%%%%%%%%%%%%%%%%%%%%%
\begin{center}
  \textsc{Sobre las consecuencias}
\end{center}

\item Si \textbf{incumplo} alguna de estas normas \textbf{perderé el
    derecho} a usar el móvil de forma temporal \textbf{o permanente}.

\item Cuando incumpla una de estas normas y mis padres me quiten el
  móvil tengo \textbf{la obligación y el derecho a hablar de lo ocurrido}.

\end{enumerate}

\end{document}

%%% Local Variables:
%%% mode: latex
%%% TeX-master: t
%%% TeX-PDF-mode: t
%%% ispell-local-dictionary: "castellano"
%%% End:
